\documentclass[12pt]{article}
\usepackage{listings}
\usepackage{xcolor}

% Customizing code blocks
\lstset{
    language=Python,
    basicstyle=\ttfamily\small,
    keywordstyle=\color{blue},
    stringstyle=\color{red},
    commentstyle=\color{gray},
    showstringspaces=false,
    frame=single,
    breaklines=true
}

\title{RPC File Transfer}
\author{Nguyen Duc Tam - 22BI13400}
\date{}

\begin{document}

\maketitle

\section*{1. Protocol Design}

The protocol is designed for file transfer using a client-server model with \textbf{Remote Procedure Calls (RPC)}.

\begin{verbatim}
        +-----------------+           +-----------------+
        |     Client      |           |     Server      |
        +-----------------+           +-----------------+
                |                             |
                |      Connect via RPC        |
                |---------------------------->|
                |                             |
                |    Invoke save_file RPC     |
                |---------------------------->|
                |                             |
                |         Acknowledge         |
                |<----------------------------|
\end{verbatim}

The client connects to the server, encodes the file in Base64, and sends it to the server by calling a remote method. The server decodes and saves the file locally.

\section*{2. System Organization}

The system consists of two components:
\begin{itemize}
    \item \textbf{Server:} Provides an RPC endpoint that allows the client to invoke a method for sending and saving a file.
    \item \textbf{Client:} Encodes the file and uses RPC to send the file data to the server.
\end{itemize}

\begin{verbatim}
            +-----------------+           +-----------------+
            |     Client      |           |     Server      |
            +-----------------+           +-----------------+
                    |                             |
                    |  Encodes and Sends File     |
                    |---------------------------->|
                    |                             |
                    |       Decodes & Saves       |
                    |<----------------------------|
\end{verbatim}

\section*{3. Implementation of File Transfer}

The file transfer is implemented using Python's \texttt{xmlrpc} library. 

\subsection*{Server Code}
\begin{lstlisting}
from xmlrpc.server import SimpleXMLRPCServer
import base64

# Set up
host = "0.0.0.0"
port = 12345
output_file = "file_received.txt"

# Function to receive file
def save_file(encoded_data):
    with open(output_file, "wb") as file:
        file.write(base64.b64decode(encoded_data))
    return "File received"

# Initialize server
server = SimpleXMLRPCServer((host, port))
print(f"RPC Server is running on {host}:{port}...")
server.register_function(save_file, "save_file")

# Run the server
server.serve_forever()
\end{lstlisting}

\subsection*{Client Code}
\begin{lstlisting}
import xmlrpc.client
import base64

# Set up
host = "http://0.0.0.0"
port = 12345
server_url = f"{host}:{port}"
filename = input("Input filename: ")

# Initialize client
client = xmlrpc.client.ServerProxy(server_url)

# Encode file data
with open(filename, "rb") as file:
    encoded_data = base64.b64encode(file.read()).decode()

# Send file to server
response = client.save_file(encoded_data)
print(response)
\end{lstlisting}

\section*{4. Advantages of RPC-Based Design}

The RPC-based design improves upon the socket-based approach by:
\begin{itemize}
    \item Abstracting the communication protocol, making it easier to add more remote methods.
    \item Enabling a more structured and modular approach to server-client interaction.
    \item Providing cross-language support for better interoperability.
\end{itemize}

\section*{5. Task Distribution}

The task is handled individually and locally.

\end{document}
