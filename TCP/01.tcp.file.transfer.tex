\documentclass[12pt]{article}
\usepackage{listings}
\usepackage{xcolor}

% Customizing code blocks
\lstset{
    language=Python,
    basicstyle=\ttfamily\small,
    keywordstyle=\color{blue},
    stringstyle=\color{red},
    commentstyle=\color{gray},
    showstringspaces=false,
    frame=single,
    breaklines=true
}

\title{TCP File Transfer}
\author{Nguyen Duc Tam - 22BI13400}
\date{}

\begin{document}

\maketitle

\section*{1. Protocol Design}

The protocol was designed for a simple file transfer using a client-server model using socket.

\begin{verbatim}
        +-----------------+           +-----------------+
        |     Client      |           |     Server      |
        +-----------------+           +-----------------+
                |                             |
                |       Connect to Server     |
                |---------------------------->|
                |                             |
                |         File Transfer       |
                |---------------------------->|
                |                             |
                |          Acknowledge        |
                |<----------------------------|
\end{verbatim}

The client initiates a connection to the server, sends the file data in chunks, and closes the connection.

\section*{2. System Organization}

The system consists of two components:
\begin{itemize}
    \item \textbf{Server:} Waits for a connection, receives file data from the client, and saves it locally.
    \item \textbf{Client:} Connects to the server and sends the file.
\end{itemize}

\begin{verbatim}
            +-----------------+           +-----------------+
            |     Client      |           |     Server      |
            +-----------------+           +-----------------+
                    |                             |
                    |        Sends File Data      |
                    |---------------------------->|
                    |                             |
                    |          Saves File         |
                    |<----------------------------|
\end{verbatim}

\section*{3. Implementation of File Transfer}

The file transfer is implemented using Python's \texttt{socket} library. 

\subsection*{Server Code}
\begin{lstlisting}
import socket

# Set up
host="0.0.0.0"
port=12345
buffer_size=1024
output_file="file_received.txt"

# Initialize server
server_socket = socket.socket(socket.AF_INET, socket.SOCK_STREAM)
server_socket.bind((host, port))
server_socket.listen(1)

print(f"Server is waiting on {host}:{port}...")

# Connect with client
client_connection, client_address = server_socket.accept()
print(f"Connected with {client_address}")

# Receive file
with open(output_file, "wb") as file:
    while True:
        data = client_connection.recv(buffer_size)
        if not data:
            break
        file.write(data)

print("File received")

# Close connection and server
client_connection.close()
server_socket.close()
\end{lstlisting}

\subsection*{Client Code}
\begin{lstlisting}
import socket

# Set up
filename = input("Input filename: ")
host="0.0.0.0"
port=12345
buffer_size=1024

# Initialize connection with server
client_socket = socket.socket(socket.AF_INET, socket.SOCK_STREAM)
client_socket.connect((host, port))

print(f"Set up connection with {host}:{port}")

# Send file
with open(filename, "rb") as file:
    while True:
        data = file.read(buffer_size)
        if not data:
            break
        client_socket.sendall(data)

print("Sent successfully.")

# Close connection
client_socket.close()
\end{lstlisting}

\section*{4. Task Distribution}

The task is handled individually and locally.

\end{document}
